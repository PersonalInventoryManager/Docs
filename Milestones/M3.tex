\documentclass{article}
\usepackage{fullpage}
\usepackage{color}
\usepackage[normalem]{ulem}
\newcommand{\eric}{\textcolor{blue}{[Eric]}}
\newcommand{\richard}{\textcolor{red}{[Richard]}}
\newcommand{\taylor}{\textcolor{green}{[Taylor]}}
\newcommand{\susi}{\textcolor{cyan}{[Susi]}}
\hyphenpenalty=100000
\usepackage{graphicx}
\DeclareGraphicsExtensions{.pdf,.png,.jpg}
\begin{document}
\setlength{\voffset}{3.5in}
\title{Milestone 3}
\author{Team Sriram\\
(Susi Cisneros, Eric Henderson, Taylor Purviance and Richard Thai)}
\date{18 October 2011}
\maketitle
\clearpage
\setlength{\voffset}{0pt}
\tableofcontents
\clearpage
~\\
\begin{Large}\textbf{Changes (based off Git commits)}\end{Large}\\
~\\
\begin{tabular}{ | p{2in} | p{4.5in} | }
\hline
\textbf{Date Time} & \textbf{Description}\\
\hline
\hline
\end{tabular}
\clearpage

\section{Executive Summary}
TODO: \richard\\
NOTE: mention that for supportability requirements, writing good code should take precedence over logging/tracing routes for errors

\section{Introduction}
TODO: \richard

\section{Project Background}

\section{Usability Requirements}
\begin{itemize}
\item User should be able to familiarize himself or herself with the software within one hour
\item The client will judge the usability of the program
\item An online help system/documentation will not be required
\end{itemize}

\section{Performance Requirements}
\begin{itemize}
\item Searches should take no longer than 3 seconds
\item System should handle 1000 entries about as quickly as 10 entries
\item If the system becomes degraded, an error should be thrown or the system should go offline until client can restart it
\end{itemize}

\section{Reliability Requirements}
\begin{itemize}
\item System should not be down more than once every 2 months
\item System will be repaired when the client gets around to it
\item The system must not have data corruption (100\% accuracy)
\item No more than 2 bugs per thousand lines of code
\item Minimize the number of critical bugs, but they are still allowed
\item Overly complex operations should be split up into several simpler operations
\end{itemize}

\section{Supportability Requirements}
\begin{itemize}
\item Should be able to be supported by open source community and the client
\item The source code needs to be kept clean and easy to understand
\item The source code will follow Ruby coding standards and naming conventions
\item The client should be able to extend the system with features of moderate complexity within four hours; this is an intended effect from following the Ruby coding standards
\item The source code should be as modular as possible in order to simplify extension
\item Adding/modifying an API route should as difficult as adding/modifying a web application feature of equal complexity
\end{itemize}

\section{Hardware and Software Interfaces}
\begin{itemize}
\item Ask Tim if there are any hardware interfaces (how much do we need to worry about UPC scanners)
\item The product will be designed for compatibility with Chrome and Firefox browsers
\item The product will utilize SQLite
\end{itemize}

\section{Documentation, Installation, Legal, and Licensing Requirements}
\begin{itemize}
\item The project should be open source distributed under the BSD license
\item Copyright ownership should be given to the client
\item Installation should only require installing the necessary packages and pulling the code from the Git repository into the web server directory
\item Only minimal documentation is required; other than a readme, it is mostly just necessary to document complex sections of code
\end{itemize}

\section{Design Constraints}
\begin{itemize}
\item The project must be open source
\item The project must be completed within five months
\item The project must be completed without funding from the client
\item The project must utilize REST, not SOAP
\item Develop in Ruby utilizing Sinatra for web framework
\item Accessible via the web.
\end{itemize} 

\section{User Interfaces}

\section{Index and Glossary}

\section{References}

\end{document}