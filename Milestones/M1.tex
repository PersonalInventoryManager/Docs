\documentclass{article}
\usepackage{fullpage}
\usepackage{color}
\usepackage[normalem]{ulem}
\newcommand{\eric}{\textcolor{blue}{[Eric]}}
\newcommand{\richard}{\textcolor{red}{[Richard]}}
\newcommand{\taylor}{\textcolor{green}{[Taylor]}}
\newcommand{\susi}{\textcolor{cyan}{[Susi]}}
\hyphenpenalty=100000
\begin{document}
\setlength{\voffset}{3.5in}
\title{Milestone 1}
\author{Team Sriram\\
(Susi Cisneros, Eric Henderson, Taylor Purviance and Richard Thai)}
\date{23 September 2011}
\maketitle
\clearpage
\setlength{\voffset}{0pt}
\tableofcontents
\clearpage
~\\
\begin{Large}\textbf{Changes (based off GIT commits)}\end{Large}\\
~\\
Note: The committers listed are the people who made the commit, and are not necessarily an accurate depiction of contributions to the milestone.\\
~\\
\begin{tabular}{ | p{3.75in} | p{1.5in} | p{1.25in} | }
\hline
\textbf{Description} & \textbf{Date} & \textbf{Committer}\\
\hline
\hline
Document started & 15 September 2011 & Richard Thai\\
\hline
Initial work on document & 15 September 2011 & Eric Henderson\\
\hline
Some formatting added to document & 15 September 2011 & Eric Henderson\\
\hline
Formatting work continued, Table of Contents added & 15 September 2011 & Eric Henderson\\
\hline
Fixed some table formatting issues & 15 September 2011 & Eric Henderson\\
\hline
Vertically center title page & 15 September 2011 & Eric Henderson\\
\hline
Reformatted code & 15 September 2011 & Eric Henderson\\
\hline
Added some more content & 16 September 2011 & Eric Henderson\\
\hline
Added initial requirements list & 16 September 2011 & Eric Henderson\\
\hline
Added executive summary and updated requirements/needs & 18 September 2011 & Richard Thai\\
\hline
Changed formatting, sectioning, and wording & 20 September 2011 & Eric Henderson\\
\hline
Changed date format/value & 20 September 2011 & Eric Henderson\\
\hline
Working on improving executive summary and introduction & 21 September 2011 & Richard Thai\\
\hline
Committed for peer reviews & 21 September 2011 & Richard Thai\\
\hline
Made commands for marking editing/proofreading changes & 21 September 2011 & Eric Henderson\\
\hline
Added initial features/needs matrix & 21 September 2011 & Eric Henderson\\
\hline
Finished my proofreading & 21 September 2011 & Eric Henderson\\
\hline
Initial work on finalizing document & 22 September 2011 & Eric Henderson\\
\hline
Document mostly finalized & 22 September 2011 & Richard Thai\\
\hline
Document finalized except for now & 22 September 2011 & Richard Thai\\
\hline
Made some changes after proofreading, fixed table of contents & 23 September 2011 & Eric Henderson\\
\hline
Added needs table & 23 September 2011 & Susi Cisneros\\
\hline
Added this change history & 23 September 2011 & Eric Henderson\\
\hline
\end{tabular}
\clearpage
\section{Executive Summary}
This milestone, the first of a series, documents the context for a software project proposed by the client: Tim Ekl. The primary stakeholders are Team Sriram (the software developers and documentors), Sriram Mohan (the course instructor), and Tim Ekl.  The software solution will eventually be open-sourced which means that external developers and end-users will be involved as stakeholders, however these parties can be ignored for the time being. The current issue raised by the client is the inability to easily and consistently locate personal belongings and hardware. This milestone will cover the background of the client, their current system, involved parties, an overview of the product and costs, features of the product, and constraints to the solution; the solution is intended as a simplified asset-tracker scaled for personal usage. This document will provide utility for the next milestone which will focus on the project background, use cases, feature-to-use-case mapping, and Data Flow Diagrams.

\section{Introduction}
Our client, Tim Ekl, would like a system to keep track of personal belongings. Tim's profession and interests with computers have caused him to accumulate a large collection of computer hardware. Currently, Tim maintains a crude organization system which involves labeling containers in order to categorize the different areas to place objects--whether it be for storage or extraction. However, the current system is not as effective as the client would like since there have been multiple instances in which items were not found when they were needed and replaced unecessarily, resulting in lost capital. In addition, the current system's generic labeling is not a sure-fire method for finding items; they are only meant to guide Tim and help him make educated guesses concerning the general location of his items. This results in lost time looking for items which may not even exist.\\
The client wishes to replace the current system with one that should be comparable to corporate asset-tracking systems with the capability to tag assets with a wide variety of attributes, though more streamlined since it is intended for use on a personal scale. Since this project is an initiative from a college program, there will be no monetary capital invested towards development. However, it is expected that each developer involved in the project will invest aproximately five hours weekly towards the project; this results in roughly 340 man hours in terms of a time budget for the project. In addition, all developers will respect the client's preference to construct the product utilizing Ruby (programming language) and Sinatra (web framework). There exist more requirements and constraints which will be elaborated on later in this document. Ultimately the product will be a web application backed by a database.

\section{Client Background}
Tim Ekl is a Rose-Hulman graduate student who possesses a significant amount of computer hardware.  He plans on using this system to be able to quickly and easily locate the equipment he wants to use.  Tim is an experienced developer and plans on maintaining the system after it is finished.  

\section{Current System}
The client does not have a software solution in place.  Currently, Tim has a primitive categorization system in place which involves labeling boxes and then trying to deduce the location of a desired component.  The current system poses a few issues such as not always allowing him to find his items, i.e. there have been instances where an item was found after capital was spent to replace it.

\section{User/Stakeholder Description}

\subsection{User Profile}
The client will be the main user of the system. While the client is very familiar with sophisticated technologies and software, the final product will be open-source and will be exposed to many more users who will have a broad spectrum of technical proficiency--so the final product will be made with these prospective end users in mind. The client is an accumulator of hardware and disc media, having approximately 1,000 such items which he wants catalogued. Ultimately, identification of the client's items will be based off of a unique identifier, a category, and as many attributes as necessary to describe it.

\subsection{Stakeholder Profiles}

\subsubsection{Stakeholder: Tim Ekl}
\textbf{Role:} Client\\
\textbf{Success:} The final product reasonably satisfies the needs and requirements as defined.\\
\textbf{Failure:} The final product is not easily usable and cannot be extended without major reworking.

\subsubsection{Stakeholder: Team Sriram}
\textbf{Role:} Requirements, Specifications, and Development Team\\
\textbf{Success:} The final product has its requirements defined and prioritized with a satisfactory amount of the high priority features implemented.  The client is able to easily extend and work with the source code provided.\\
\textbf{Failure:} The final product delivered fails to successfully implement the critical features requested by the client.

\subsubsection{Stakeholder: Sriram Mohan}
\textbf{Role:} Class Instructor\\
\textbf{Success:} Team Sriram understood the concepts and processes utilized during the course of the project. The client is satisfied with the final product.\\
\textbf{Failure:} Team Sriram was unable to deduce any concepts or processes discussed during the course. The client is not satisfied with the final product.

\subsection{User Environment}
\begin{itemize}
\item The client uses Chrome whenever possible and prefers that development support Chrome and Firefox browsers.
\item The final product ought to operate on a Linux server with standard programming languages, programming frameworks, and Apache.  Additional packages can be installed if necessary.
\end{itemize}

\subsection{User Needs}
\begin{tabular}{ | p{0.15in} | p{4.0in} | p{.75in} |}
\hline
\textbf{ID} & \textbf{Need} & \textbf{Priority} \\
\hline
\hline
N0 & Search for parts based off their attributes & Primary \\
\hline
N1 & Compatible with a UNIX architecture & Primary \\
\hline
N2 & Identify items via bar codes & Primary \\
\hline
N3 & Keep track of the data associated with an asset & Primary \\
\hline
N4 & Organize search results & Primary \\
\hline
N5 & Insert objects in the system at any point; do not freeze the database & Secondary \\
\hline
N6 & Modify objects; including adding notes to the objects & Secondary \\
\hline
N7 & Accessible from any computer & Optional \\
\hline
N8 & View most-recently acquired asset(s) & Optional \\
\hline
N9 & View a summary of inventory data & Optional \\
\hline
\end{tabular}

\subsection{Alternatives and Competition}
Tim has researched alternative solutions but has not been able to find one that fits his needs.  Alternative systems have been intended for corporate use, are overly complicated, and are designed for large scale use.  Other issues associated with these products include incompatibility with non-Windows environments, excessive cost, and locally restricted access (not accessible through the Internet).  The client has also considered making a system for himself, however he has never had the time to implement it.

\section{Product Overview}

\subsection{Product Perspective}
The software will be an independent system to be hosted on the client's servers.  The product will replace the current solution by electronically keeping track of asset locations. In addition, the software will maintain data associated with each asset.

\subsection{Elevator Statement}
Everyone accumulates items that they currently have no use for, but do not want to dispose of.  Given enough time, a person will accumulate more items than they will be able to consciously keep track of. We propose a web application which will allow for simple and streamlined asset tracking. The software will maintain an organizied and systematic inventory which will also keep track of each item's location.

\subsection{Summary of Capabilities}
\begin{tabular}{ | p{4.0in} | p{2.0in} | }
\hline
\textbf{Customer Benefit} & \textbf{Supporting Features}\\
\hline
\hline
Expediently track items based off of specific criteria & F1, F4, F5, F6, F8\\
\hline
View statistics of asset inventory & F4, F7\\
\hline
Keep all data in one place & F0, F2\\
\hline
Access data from anywhere with Internet access & F0\\
\hline
Compatible with the client's current user environment & F9\\
\hline
Change data associated with items & F3\\
\hline
\end{tabular}

\subsection{Assumptions and Dependencies}
\begin{itemize}
\item Availability of network/web hosting.
\item Availability of Ruby and Sinatra framework to run the software.
\item Operates on the Unix operating system.
\end{itemize}

\subsection{Rough Cost Estimate}
Since the solution will be the product of a course project, no monetary capital will be given nor required.  Concerning a time budget, 340 man hours will be invested into the final product (20 hours per week given four team members).

\section{Non-functional Requirements}

\subsection{General}
\begin{itemize}
\item Responsive on the local network.
\item Local/client storage on web browsers using SQLite and Sinatra.
\item Allow for easy transfer of inventory data; involves a simple copying, tar'ing, moving, and untar'ing of the directory.
\item Avoid crashing the application or at least handle the crash gracefully; avoid hard server restarts or corruptions.
\end{itemize}

\subsection{Testing}
\begin{itemize}
\item Cucumber is the preferred testing framework. RSpec is recommended if Cucumber is not used.
\item Manually coded unit tests should be used at minimum if no testing framework is utilized.
\end{itemize}

\section{Features}
\subsection{Feature Listing}
\begin{tabular}{ | p{0.15in} | p{2.0in} | p{0.5in} | p{0.5in} | p{0.5in} | p{0.6in} | p{0.5in} | p{0.65in} | }
\hline
\textbf{ID} & \textbf{Feature} & \textbf{Priority} & \textbf{Effort} & \textbf{Risk} & \textbf{Stability} & \textbf{Target Release} & \textbf{Assigned To} \\
\hline
\hline
F0 & Web-accesible with an online API & Critical & High & High & Low & 1.0 & Unassigned \\
\hline
F1 & Search for matching attribute (such as a bar code, text description, or unified search); have simple and advanced searches & Critical & High & High & Low & 1.0 & Unassigned \\
\hline
F2 & Add assets to the inventory & Critical & Low & High & Low & 1.0 & Unassigned \\
\hline
F3 & Modify assets in the inventory & Critical & Low & High & Low & 1.0 & Unassigned \\
\hline
F4 & Each category of an asset has a related set of optional attributes & Critical & Low & High & Low & 1.0 & Unassigned \\
\hline
F5 & Use a UPC-A barcode as the unique identifier for each asset & Critical & Low & Medium & Low & 1.0 & Unassigned \\
\hline
F6 & Provide an updated list of recently-added assets & Useful & Medium & Low & Medium & 1.5 & Unassigned \\
\hline
F7 & Generate reports of asset inventory & Useful & High & Low & Medium & 2.0 & Unassigned \\
\hline
F8 & Sort search results based off of barcode, title, and modified/created timestamp & Useful & High & Low & Low & 2.0 & Unassigned \\
\hline
F9 & Deployable on a Linux-based operating system & Critical & Low & High & Low & 1.0 & Team Sriram \\
\hline
\end{tabular}\\
~\\
~\\
\subsection{Feature-to-Need Correspondence}
\begin{tabular}{ | c || c | c | c | c | c | c | c | c | c | c | }
\hline
   & N0 & N1 & N2 & N3 & N4 & N5 & N6 & N7 & N8 & N9 \\
\hline
\hline
F0 &    &    &    &    &    &    &    & X  &    &    \\
\hline
F1 & X  &    & X  &    &    &    &    &    &    &    \\
\hline
F2 &    &    &    & X  &    & X  &    &    &    &    \\
\hline
F3 &    &    &    & X  &    &    & X  &    &    &    \\
\hline
F4 &    &    &    & X  &    & X  & X  &    &    &    \\
\hline
F5 & X  &    & X  & X  &    &    &    &    &    &    \\
\hline
F6 &    &    &    &    &    &    &    &    & X  &    \\
\hline
F7 &    &    &    &    &    &    &    &    &    & X  \\
\hline
F8 & X  &    & X  &    & X  &    &    &    &    &    \\
\hline
F9 &    & X  &    &    &    &    &    &    &    &    \\
\hline
\end{tabular}

\section{Constraints}
\begin{tabular}{ | p{0.15in} | p{0.75in} | p{1.8in} | p{2.3in} | p{0.5in} | }
\hline
\textbf{ID} & \textbf{Source} & \textbf{Constraint} & \textbf{Rationale} & \textbf{Cost}\\
\hline
\hline
C0 & Development & The project must be open source. & The client wants to be able to extend the project's functionality as well as allow third parties to benefit from the software source code. & Low \\
\hline
C1 & Time & The project must be completed within five months. & This is the duration of the Junior Project sequence for Team Sriram. & High \\
\hline
C2 & Budget & The project must be completed without funding from the client. & This is an academic undertaking and is not for monetary gain. & Low \\
\hline
C3 & Software Architecture & The project must utilize REST, not SOAP. & The client strongly prefers not to use SOAP based off of past experience. & Low \\
\hline
C4 & Technology & Develop in Ruby utilizing Sinatra for web framework. & The client prefers the development language and framework to be familiar to him for further extension. & Medium \\
\hline
\end{tabular}

\section{Index and Glossary}
\textbf{Assigned to}: Developer ultimately responsible for the implementation of a feature (6).\\ \\
\textbf{Constraint}: A limitation or bound for the development of the project (7).\\ \\
\textbf{Cost}: How big of a negative impact a constraint will have on the development process (7).\\ \\
\textbf{Effort}: Expectation of the resources and time consumed for a feature (6).\\ \\
\textbf{Feature}: A system capability that fulfills a user need (6).\\ \\
\textbf{Priority}: Description of how essential a feature is to the project (6).\\ \\
\textbf{Risk}: Probability that a feature will instigate delays in the project (6).\\ \\
\textbf{Source}: The area that a constraint is tied to (7).\\ \\
\textbf{Stability}: Likelihood that the understanding of a feature will change (6).\\ \\
\textbf{Target Release}: Expected release iteration of a testable feature (6).

\section{References}
\hangindent=1.4cm
\textbf{(1)} Leffingwell, Dean, and Don Widrig.
\emph{Managing Software Requirements: a Use Case Approach}.
Addison-Wesley, Boston,
2nd Edition,
2003.

\end{document}