\documentclass{article}
\usepackage{fullpage}
\hyphenpenalty=100000
\begin{document}
\title{Milestone 1}
\author{Team Sriram (Susi Cisneros, Eric Henderson, Taylor Purviance and Richard Thai)}
\date{September 15, 2011}
\maketitle
\clearpage
\tableofcontents
\clearpage
\section{Executive Summary}
TODO\\
\section{Introduction}
Our client, Tim Ekl, would like a system to keep track of personal belongings. Tim’s profession and interests with computers has caused him to collect a large amount of computer hardware. The system ought to be comparable to corporate asset-tracking systems with the capability to tag a wide variety of assets, though simpler since the product should be on a personal scale. Ultimately the product will be a web application backed by a database.

\section{Client Background}
Tim Ekl is a Rose-Hulman graduate student who possesses a good amount of computer hardware.  He plans on using this system to be able to find the equipment he wants to use as he builds or upgrades a computer easily and quickly.  Tim is an experienced developer and plans on maintaining the system after it is finished.  

\section{Current System}
The client currently does not have a software solution in place.  Currently, Tim has a primitive categorization system in place which involves labeling boxes which he deduces the location of a possible item.  The current system has a few issues as it does not always allow him to find his items, i.e. there has been instances where an item was found after capital was spent to replace it.
\section{User/Stakeholder Description}
\section{User Profile}
The client will be the main user of the system.  While the client is very familiar with sophisticated, the final product will be open-sourced and will be exposed to many more users who will have a broad spectrum of technical proficiency--so the final product will be made with these possible end users in mind.
TO ASK: Do we need to keep track of key responsibilities, deliverables, trends that make the job easier/harder, problems that interfere with success, and definition of success + how they benefit from it?
\section{Stakeholder Profiles}
\subsection{Stakeholder: Team Sriram}
\textbf{Role:} Development team involved in the planning of the final project.\\
\textbf{Success:} The final product has its requirements defined and prioritized and a satisfactory amount of the high priority features are implemented.  The client is able to easily extend and work with the source code provided.\\
\textbf{Failure:} The final product does not satisfy a reasonable amount of the requirements defined.

\subsection{Stakeholder: Sriram Mohan}
\textbf{Role:} Class Instructor\\
\textbf{Success:} The team involved understood the concepts and processes utilized during the course of the project and will remember them for future projects. The client is satisfied with the final product.\\
\textbf{Failure:} The team was unable to deduce any concepts or processes discussed during the course. The client is not satisfied with the final product.

\subsection{Stakeholder: Tim Ekl}
\textbf{Role:} The client who defines the requirements and criteria for the project.\\
\textbf{Success:} The final product reasonably satisfies his needs and requirements. (This might have some redundancy)\\
\textbf{Failure:} The final product is not easily usable and cannot be extended without major reworking.
\section{User Environment}
TODO\\
Ask what system he uses, how long he usually searches for something before giving up, how long he expects to search for something w/ the new system
\section{User Needs}
TODO
\section{Alternatives and Competition}
Tim has researched alternative solutions but has not been able to find one that fits his needs.  The other systems have all been for corporate use and have been overly complicated and for too large of a scale.  Other issues that he had with these products were that they were only available on windows, expensive, and were not web-based.  Tim has also considered making a system himself, but has never had the time to implement it.

\section{Product Overview}
\subsection{Product Perspective}
This system will be an independent system, hosted on one of Tim’s servers.  This system will use the current solution as a way to physically organize the items and will improve the current solution by keeping track of exactly where a specific piece of hardware is.
\subsection{Elevator Statement}
Everyone accumulates items that they currently have no use for, but do not want to dispose of.  Given enough time, a person will accumulate more items than they will be able to consciously keep track of. We propose a web application which will easily allow for simple asset tagging and recall.

\subsection{Summary of Capabilities}
\begin{tabular}{ll}
Customer Benefit & Supporting Features\\
Increased organization of inventory & Indexing of possessed items; searchable database of items; logging of items location and state\\
Ease of information access & Web-based system; centralized database
\end{tabular}

\subsection{Assumptions and Dependencies}
\begin{itemize}
\item Network / web hosting will be available
\item Ruby Sinatra framework will be available to run solution
\end{itemize}

\subsection{Rough Estimate of the Cost}
Since the solution will be the product of a course project, no capital will be given nor required.  As for calculating a time budget, given four team members as well as the project spanning over two courses, 1280 man hours will be invested into the final product (16 hours per week).
\subsection{Feature Listing}
\begin{tabular}{ | l | l | l | l | l | l | }
\hline
ID & Feature & Need & Votes & Effort & Risk\\
\hline
0 & Open source & Need 0 & 13 & low & low\\
\hline
\end{tabular}
\subsection{Constraints}
\begin{tabular}{ | l | l | l | }
\hline
Source & Constraint & Rational\\
\hline
Development & The project must be open source. & The ethos of the client and his desire to be able to extend the project’s functionality easily.\\
\hline
Time & The project must be completed in 1 term (NOTE: verify this) & After this period, the Requirements and Specifications class will be over and the team will be disbanded\\
\hline
Budget & The project must be completed without funding from the client. & This is an academic undertaking and is not for monetary gain.\\
\hline
Operating System & The solution must be able to be run / hosted in a Windows-less environment. & The client’s existing hosting is Linux / Mac and there will be no addition of a Windows server\\
\hline
\end{tabular}

\section{Index and Glossary}

\section{References}

\end{document}