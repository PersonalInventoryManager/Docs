\documentclass{article}
\usepackage{fullpage}
\usepackage{color}
\usepackage[normalem]{ulem}
\usepackage{hyperref}
\hypersetup{colorlinks}
\newcommand{\eric}{\textcolor{blue}{[Eric]}}
\newcommand{\richard}{\textcolor{red}{[Richard]}}
\newcommand{\taylor}{\textcolor{green}{[Taylor]}}
\newcommand{\susi}{\textcolor{cyan}{[Susi]}}
\hyphenpenalty=100000
\usepackage{graphicx}
\DeclareGraphicsExtensions{.pdf,.png,.jpg}
\begin{document}
\setlength{\voffset}{3.5in}
\title{Milestone 5}
\author{Team Sriram / Team 15\\
(Susi Cisneros, Eric Henderson, Taylor Purviance and Richard Thai)}
\date{10 February 2012}
\maketitle
\clearpage
\setlength{\voffset}{0pt}
\tableofcontents
\clearpage
~\\
\begin{Large}\textbf{Changes (based off Git commits)}\end{Large}\\
~\\
\begin{tabular}{ | p{2in} | p{4.5in} | }
\hline
\textbf{Date Time} & \textbf{Description}\\
\hline
\hline
9 February 2012 11:10 am & Initial Milestone 5 created\\
\hline
\end{tabular}

\clearpage
\section{Executive Summary}
not necessary, but a good idea.

\section{Introdiction}
foo.

\section{Analysis Models}
foo.

\section{Logical Architecture}
foo.

\section{Design}
foo.

\section{Integration and Acceptance Test Plan}
copy this from the compilated milestone.

\section{Who done what}
make this in word, and bring the pdf over.

\section{Index and Glossary}
\textbf{Assigned to}: Developer ultimately responsible for the implementation of a feature (\pageref{feature}).\\ \\
\textbf{Effort}: Expectation of the resources and time consumed for a feature (\pageref{feature}).\\ \\
\textbf{Exact Matching}: Valid search candidates are determined by strict equality with the search input(s) (\pageref{feature}).\\ \\
\textbf{Feature}: A system capability that fulfills a user need (\pageref{feature}).\\ \\
\textbf{Flow}: A particular series of ordered steps through a sequence of events in a use case (\pageref{flow}).\\ \\
\textbf{Fuzzy Matching}: Valid search candidates are determined based on similarity to the original search input(s) (\pageref{feature}).\\ \\
\textbf{Post-condition}: An expectation of the system's state after the events in a use case occur (\pageref{post_cond}).\\ \\
\textbf{Pre-condition}: An assumption about the state of a system before a use case's events are followed (\pageref{pre_cond}).\\ \\
\textbf{Priority}: Description of how essential a feature is to the project (\pageref{feature}).\\ \\
\textbf{REST}: Stands for ``Representational state transfer;'' is a style software architecture often used in servers for simplicity and scalability. Amongst other aspects of REST, it is stateless. This means each request sent from the client of the server that uses REST to the server does not require any information from any previous requests. All information required to fulfill the request must be sent with the request (\pageref{rest}).\\ \\
\textbf{Risk}: Probability that a feature will instigate delays in the project (\pageref{feature}).\\ \\
\textbf{Scenario}: A possible permutation of a set of flows that lead from one flow to another (\pageref{scenario}).\\ \\
\textbf{Stability}: Likelihood that the understanding of a feature will change (\pageref{feature}).\\ \\
\textbf{Target Release}: Expected release iteration of a testable feature (\pageref{feature}).\\ \\
\textbf{Test Case}: Specifies what the expected result of a series of actions will have in a system (\pageref{test_case}).\\ \\
\textbf{Universal Product Code (UPC)}: a specific kind of barcode assign uniquely to each item in the server (\pageref{upc}).\\ \\
\textbf{Use case}: A description of the steps performed by a user and system that leads the user towards a useful outcome (\pageref{use_case}).\\ \\
\textbf{Wildcard Matching}: Valid search candidates are determined based on adherence to the user's given search pattern (\pageref{feature}).\\ \\

\section{References}
\hangindent=1.4cm
\textbf{(1)} Leffingwell, Dean, and Don Widrig.
\emph{Managing Software Requirements: a Use Case Approach}.
Addison-Wesley, Boston,
2nd Edition,
2003.\\

\noindent\hangindent=1.4cm
\textbf{(2)} ``Ruby 1.9.2''
\emph{Download Ruby.} Ruby. Web.  6 October 2011. \\

\noindent\hangindent=1.4cm
\textbf{(3)} ``Sinatra 1.3.0''
\emph{Sinatra: Documentation.} Sinatra. Web.  6 October 2011.\\

\noindent\hangindent=1.4cm
\textbf{(4)} ``Ubuntu 11.04''
\emph{Download Ubuntu.} Ubuntu. Web.  6 October 2011.\\

\noindent\hangindent=1.4cm
\textbf{(5)} ``SQLite 3.7.8''
\emph{SQLite Download Page.} SQLite. Web.  6 October 2011.\\

\noindent\hangindent=1.4cm
\textbf{(6)} ``Cucumber 1.1.0''
\emph{cucumber/cucumber.} Cucumber. Web.  6 October 2011.\\

\noindent\hangindent=1.4cm
\textbf{(7)} ``RSpec 2.6.0''
\emph{RSpec Documentation.} Relish. Web.  6 October 2011.\\

\noindent\hangindent=1.4cm
\textbf{(8)} ``DataMapper 1.1.0''
\emph{DataMapper - Documentation.} DataMapper. Web.  6 October 2011.\\

\noindent\hangindent=1.4cm
\textbf{(9)} ``Google Chrome 14.0.8'' 
\emph{About Google Chrome.} Google. Web.  7 October 2011.\\

\noindent\hangindent=1.4cm
\textbf{(10)} ``Firefox 7.0.1''
\emph{Mozilla Firefox Web Browser - Free Download.} Mozilla. Web.  7 October 2011.\\

\noindent\hangindent=1.4cm
\textbf{(11)} ``Apache 2.2''
\emph{Apache HTTP Server Version 2.2 Documentation - Apache HTTP Server.} Apache. Web.  7 October 2011.\\

\noindent\hangindent=1.4cm
\textbf{(12)} ``BSD''
\emph{The FreeBSD Copyright.} The FreeBSD Project. Web. 18 October 2011.\\

\noindent\hangindent=1.4cm
\textbf{(13)} ``GitHub''
\emph{GitHub - Social Coding.} GitHub, Inc. Web. 28 October 2011.\\

\noindent\hangindent=1.4cm
\textbf{(14)} ``The Unofficial Ruby Usage Guide'' \emph{The Unofficial Ruby Usage Guide.} Caliban. 28 October 2011.\\

\end{document}