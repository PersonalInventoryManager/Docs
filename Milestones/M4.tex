\documentclass{article}
\usepackage{fullpage}
\usepackage{color}
\usepackage[normalem]{ulem}
\newcommand{\eric}{\textcolor{blue}{[Eric]}}
\newcommand{\richard}{\textcolor{red}{[Richard]}}
\newcommand{\taylor}{\textcolor{green}{[Taylor]}}
\newcommand{\susi}{\textcolor{cyan}{[Susi]}}
\hyphenpenalty=100000
\usepackage{graphicx}
\DeclareGraphicsExtensions{.pdf,.png,.jpg}
\begin{document}
\setlength{\voffset}{3.5in}
\title{Milestone 4}
\author{Team Sriram\\
(Susi Cisneros, Eric Henderson, Taylor Purviance and Richard Thai)}
\date{28 October 2011}
\maketitle
\clearpage
\setlength{\voffset}{0pt}
\tableofcontents
\clearpage
~\\
\begin{Large}\textbf{Changes (based off Git commits)}\end{Large}\\
~\\
\begin{tabular}{ | p{2in} | p{4.5in} | }
\hline
\textbf{Date Time} & \textbf{Description}\\
\hline
\hline
\end{tabular}
\clearpage
\section{Executive Summary}
\textbf{TODO:} \richard

\section{Introduction}
\textbf{TODO:} \richard

\section{Project Background}

\section{Coding Standards}
\begin{itemize}
\item The code will use spaces, not tabs, to indent blocks
\item The code will use indentation increments of two spaces
\item The code will use underscores to separate words in variable names
\item The code will follow any other Ruby coding standards
\item The folder structure for the product will be similar to that of the GasTracker project, a folder structure the client is familiar with
\end{itemize}

\section{Change Control}

\subsection{Receiving Requests}
If the client wishes to change the code, he will notify the team at a meeting.  Otherwise, he will fork the repository, make the changes himself, and submit a pull request.

\subsection{Managing Change}
\begin{itemize}
\item Changes to sections carried over from previous milestones will be noted in a ``Deltas'' section
\item Changes to a milestone will be noted using the Git change log
\item When content carried over from a previous milestone is changed, any milestones containing that content should be updated
\item Changes to anything in the repository will be noted with a detailed commit message
\end{itemize}

\subsection{Source Control}
\begin{itemize}
\item The source will be kept in a public repository on GitHub
\item Changes to the source will be noted with a detailed commit message
\end{itemize}

\section{Test Cases}

\subsection{Add an Item}
\paragraph{Description}
~\\ \\
\textbf{NOTE:} Alternate Flow A0 will not be tested, since the server going offline is not an attribute of the system that can be measured.

\paragraph{Scenario Matrix}~\\ \\
\begin{tabular}{ c  c  c  c }
\hline
Scenario ID & Originating Flow & Alternate Flow & Next Alternate\\
\hline
\hline
S0 & Basic &  & \\
\hline
S1 & Basic & A1 & \\
\hline
S2 & Basic & A2 & \\
\hline
S3 & Basic & A1 & A2\\
\hline
\end{tabular}\\
~\\
~\\
\paragraph{Testing Matrix}~\\ \\
\begin{tabular}{ p{0.5in}  p{2.4in}  p{0.4in}  p{0.7in}  p{2.5in} }
\hline
Test Case ID & Description & Name & UPC & Expected Result\\
\hline
\hline
T0 & Basic Flow & Present & Unique & The item is added successfully and the user is redirected to the main page with an updated Recent Changes list\\
\hline
T1 & Alternate Flow A1 & Absent & Unique & The user is notified that a required field was omitted\\
\hline
T2 & Alternate Flow A1 & Present & Absent & The user is notified that a required field was omitted\\
\hline
T3 & Alternate Flow A1 & Absent & Absent & The user is notified that a required field was omitted\\
\hline
T4 & Alternate Flow A2 & Present & Non-unique & The user is notified that the UPC is not unique\\
\hline
T5 & Alternate Flows A1, A2 & Absent & Non-unique & The user is notified that a required field was omitted and that the UPC is not unique\\
\hline
\end{tabular}\\
~\\
~\\
\paragraph{Test Case to Scenario Matrix}~\\ \\
\begin{tabular}{ | c || c | c | c | c | }
\hline
    & S0 & S1 & S2 & S3\\
\hline
\hline
T0 & X  &    &    &    \\
\hline
T1 &    & X  &    &    \\
\hline
T2 &    & X  &    &    \\
\hline
T3 &    & X  &    &    \\
\hline
T4 &    &    & X  &    \\
\hline
T5 &    &    &    & X  \\
\hline
\end{tabular}

\subsection{Basic Search for an Item}
\paragraph{Description}


\subsection{Advanced Search for an Item}
\paragraph{Description}


\subsection{Search Result Sorting}
\paragraph{Description}


\subsection{View Details of an Item}
\paragraph{Description}


\subsection{Edit Details of an Item}
\paragraph{Description}


\subsection{Generate Inventory Report}
\paragraph{Description}


\section{Index and Glossary}
\textbf{TODO:} \taylor

\section{References}

\end{document}