\documentclass{article}
\usepackage{fullpage}
\usepackage{color}
\usepackage[normalem]{ulem}
\newcommand{\eric}{\textcolor{blue}{[Eric]}}
\newcommand{\richard}{\textcolor{red}{[Richard]}}
\newcommand{\taylor}{\textcolor{green}{[Taylor]}}
\newcommand{\susi}{\textcolor{cyan}{[Susi]}}
\hyphenpenalty=100000
\usepackage{graphicx}
\DeclareGraphicsExtensions{.pdf,.png,.jpg}
\begin{document}
\setlength{\voffset}{3.5in}
\title{Milestone 4}
\author{Team Sriram\\
(Susi Cisneros, Eric Henderson, Taylor Purviance and Richard Thai)}
\date{28 October 2011}
\maketitle
\clearpage
\setlength{\voffset}{0pt}
\tableofcontents
\clearpage
~\\
\begin{Large}\textbf{Changes (based off Git commits)}\end{Large}\\
~\\
\begin{tabular}{ | p{2in} | p{4.5in} | }
\hline
\textbf{Date Time} & \textbf{Description}\\
\hline
\hline
\end{tabular}
\clearpage
\section{Executive Summary}
This milestone, the fourth of a series, documents the context for a software project proposed by the client, Tim Ekl.  The primary purpose of this document is to define and establish the coding standards and change control policies that will be maintained throughout the entirety of the development of the product.  In addition, this document will define the test cases that will be covered to assess the robustness of the use cases of the product.  The test cases include scenario matrices to explain the flow of events being tested, testing matrices in order to explain the special conditions for each scenario, and a test-case-to-scenario mapping to designate the test cases to specific scenarios.  This document inherits  the Use-Case-to-Feature Correspondence as well as the Use Cases from the second milestone, as well as the Needs, Features, and Feature-to-Need mappings from the first milestone, and is considered to be fully-developed--though not exempt from change--as an artifact for future software development.

\section{Introduction}
This document is intended to extend the second milestone, which defined the use cases that satisfied the needs and features of the product, by describing the test cases that verify the functionality of the product.  The test cases will contain the scenarios, tests, and test-case-to-scenario mappings.  The scenarios will explain an expected path defined by the basic flow and alternative paths defined in the respective use case.  The tests will describe the characteristics of the relevant input fields or conditions and the expected outcome that will be enacted in response to such.  The test-case-to-scenario matrix describes the scope of the test cases over the scenarios described and help ensure completeness as well as traceability.\\
~\\
In addition, this document will define the coding standards to be used throughout the software implementation process as well as the policy for controlling change--both for software and artifacts.\\
~\\
Finally, this document inherits information from the second milestone:  Use-Case-to-Feature Correspondence and the Use Cases.  Furthermore, the information from this document is to be considered fully-developed--though not exempt from change--as an artifact for future software development.

\section{User Environment}
\begin{itemize}
\item The client uses Chrome [9] whenever possible and prefers that development support Chrome [9] and Firefox [10] browsers.
\item The final product should operate on a Linux server with standard programming languages, programming frameworks, and Apache[11].  Additional packages can be installed if necessary.
\end{itemize}

\section{User Needs}
\begin{tabular}{ | p{0.15in} | p{4.0in} | p{.75in} |}
\hline
\textbf{ID} & \textbf{Need} & \textbf{Priority} \\
\hline
\hline
N0 & Search for parts based off their attributes & Primary \\
\hline
N2 & Identify items via bar codes & Primary \\
\hline
N3 & Keep track of the data associated with an asset & Primary \\
\hline
N4 & Organize search results & Primary \\
\hline
N5 & Insert objects in the system at any point; do not freeze the database & Secondary \\
\hline
N6 & Modify objects; including adding notes to the objects & Secondary \\
\hline
N7 & Access from a second physical location & Optional \\
\hline
N8 & View most-recently acquired asset(s) & Optional \\
\hline
N9 & View a summary of inventory data & Optional \\
\hline
\end{tabular} \\
\textbf{Legend:} \\
Primary - Necessary to the system \\
Secondary - Important to the system \\
Optional - Would be nice to have in the system
\section{Features}
\subsection{Feature Listing}
\begin{tabular}{ | p{0.15in} | p{2.0in} | p{0.5in} | p{0.5in} | p{0.5in} | p{0.6in} | p{0.5in} | p{0.65in} | }
\hline
\textbf{ID} & \textbf{Feature}\label{feature} & \textbf{Priority}\label{priority} & \textbf{Effort}\label{effort} & \textbf{Risk}\label{risk} & \textbf{Stability}\label{stability} & \textbf{Target Release}\label{target_release} & \textbf{Assigned To}\label{assigned_to} \\
\hline
\hline
F0 & Online UI & Critical & High & High & Low & 1.0 & Eric \\
\hline
F2 & Add assets to the inventory & Critical & Low & High & Low & 1.0 & Richard \\
\hline
F3 & Modify assets in the inventory & Critical & Low & High & Low & 1.0 & Taylor \\
\hline
F4 & The system keeps track of attributes based on category & Critical & Low & High & Low & 1.0 & Susi \\
\hline
F5 & Use a UPC-A barcode as the unique identifier for each asset\label{upc} & Critical & Low & Medium & Low & 1.0 & Eric \\
\hline
F6 & Provide an updated list of recently-added assets & Useful & Medium & Low & Medium & 1.5 & Richard \\
\hline
F7 & Generate reports of asset inventory & Useful & High & Low & Medium & 2.0 & Taylor \\
\hline
F8 & Sort search results based off of barcode, title, and modified / created timestamp & Useful & High & Low & Low & 2.0 & Susi \\
\hline
F10 & Basic search for items based on name or UPC & Critical & High & High & Low & 1.0 & Eric \\
\hline
F11 & Advanced search for items based on all fields related to the item and its category & Critical & High & High & Low & 1.0 & Richard \\
\hline
F12 & Basic and Advanced searches allow the user to include wildcards in the query & Critical & High & High & Low & 1.0 & Taylor \\
\hline
F13 & Basic and Advanced searches will search first by exact / wildcard match, then by fuzzy match & Critical & High & High & Low & 1.0 & Susi \\
\hline
F14 & REST API & Critical & High & High & Low & 1.0 & Eric \\
\hline
\end{tabular}\label{rest}\\ \\ \\
\textbf{Legend:} \\
Critical - Highest importance \\
Important - Medium importance \\
Useful - Lowest importance \\
High / Medium / Low - Degree of a category \\
1.0 - First release of the system \\
1.5 - Next release of the system with significant changes \\
2.0 - Final release of the system \\
~\\
~\\
\subsection{Feature-to-Need Correspondence}
\begin{tabular}{ | c || c | c | c | c | c | c | c | c | c | c | }
\hline
    & N0 & N2 & N3 & N4 & N5 & N6 & N7 & N8 & N9 \\
\hline
\hline
F0  &    &    &    &    &    &    & X  &    &    \\
\hline
F2  &    &    & X  &    & X  &    &    &    &    \\
\hline
F3  &    &    & X  &    &    & X  &    &    &    \\
\hline
F4  &    &    & X  &    & X  & X  &    &    &    \\
\hline
F5  & X  & X  & X  &    &    &    &    &    &    \\
\hline
F6  &    &    &    &    &    &    &    & X  &    \\
\hline
F7  &    &    &    &    &    &    &    &    & X  \\
\hline
F8  & X  & X  &    & X  &    &    &    &    &    \\
\hline
F10 & X  & X  & X  & X  &    &    &    &    &    \\
\hline
F11 & X  & X  & X  & X  &    &    &    &    &    \\
\hline
F12 & X  & X  & X  & X  &    &    &    &    &    \\
\hline
F13 & X  & X  & X  & X  &    &    &    &    &    \\
\hline
F14 &    &    &    &    &    &    & X  &    &    \\
\hline
\end{tabular}

\section{Client Background}
Tim Ekl is a Rose-Hulman graduate student who possesses a significant amount of computer hardware.  He plans on using this system to be able to quickly and easily locate the equipment he wants to use.  Tim is an experienced developer and plans on maintaining the system after it is finished.  

\section{Current System}
The client does not have a software solution in place.  Currently, Tim has a primitive categorization system in place which involves labeling boxes and then trying to deduce the location of a desired component.  The current system poses a few issues such as not always allowing him to find his items, i.e. there have been instances where an item was found after capital was spent to replace it.

\section{Use Case-to-Feature Correspondence}
\begin{tabular}{ | c || c | c | c | c | c | c | c | c | c | c | c | c | c | c | }
\hline
    & F0 & F2 & F3 & F4 & F5 & F6 & F7 & F8 & F10 & F11 & F12 & F13 & F14 \\
\hline
\hline
UC1  &  X &  X &    &  X &  X &  X &    &    &    &    &    &    &    \\
\hline
UC2  &  X &    &    &    &  X &    &    &  X &  X &    &  X &  X &    \\
\hline
UC3  &  X &    &    &  X &  X &    &    &  X &    &  X &  X &  X &    \\
\hline
UC4  &  X &    &    &  X &  X &    &    &  X &    &    &    &    &    \\
\hline
UC5  &  X &    &    &  X &  X &    &    &    &    &    &    &    &    \\
\hline
UC6  &  X &    &  X &  X &  X &  X &    &    &    &    &    &    &    \\
\hline
UC7  &  X &    &    &  X &  X &    &  X &    &    &    &    &    &    \\
\hline
\end{tabular}\\ \\
NOTE: F14 (REST API feature) is not mapped to a use case because it is not a visible user feature, but still serves as a function if there is an application that attempts to access the system.

\section{Use Cases}
\label{use_case}

\paragraph{Use case syntax}
~\\
Each use case is divided into 8 sections:
\begin{itemize}
\item A ``Basic Description'' section which gives an overview of what functionality the use case demonstrates.
\item An ``Actors'' section to describe who or what interacts with the system in the use case.
\item A ``Pre-conditions'' section which contains the assumptions made, especially those pertaining to the state of the system, prior to the start of the use case.\label{pre_cond}
\item A ``Basic Flow of Events'' section which details the order of events done by the actor(s) and the system under standard conditions. If a particular step in the basic flow has the possibility of failing to occur successfully, one or more alternative flow listings are presented in brackets following the possibility of failure. Each alternative flow listing matches to an alternative flow in the ``Alternative Flow'' section. If a basic flow step fails, the alternative flow that is listed with it is followed as the next step.
\item An ``Alternative Flow of Events'' section which holds each alternative flow listed in the basic flow. Each alternative flow has a unique identifier used to reference it, a short name describing the condition that would cause the alternative flow to be followed, and the order of actions performed by the actor(s) and the system under the flow's conditions. If an action has the possibility of failing, alternatives to that flow are presented just as they are in the basic flow with a reference to the alternative flow to take.
\item A ``Post-conditions'' section which contains guarantees on the result of the use case.\label{post_cond}
\item An ``Extension Points'' section which contains references to other use cases which the completion of the current use case might lead into.
\end{itemize}

\subsection{Add an Item}

\paragraph{Brief Description}
This use case shows the procedure for adding an item to the inventory.

\paragraph{Actors}
\begin{itemize}
\item User
\end{itemize}

\paragraph{Pre-conditions}
\begin{itemize}
\item The current page must have a link to the ``Add Item'' page.
\end{itemize}

\paragraph{Basic Flow of Events}
\begin{enumerate}
\item The user clicks the ``Add Item'' link. [A0]
\item The user fills in the required fields (UPC and Name). [A1]
\item The system dynamically adds the optional attribute fields for the category chosen.
\item The user fills in the appropriate optional attributes and ``Notes'' fields.
\item The user clicks the ``Add Item'' button. [A2]
\item The system adds the item to the inventory database.
\end{enumerate}

\paragraph{Alternative Flows of Events}

\subparagraph{Alternative Flow A0: Server is Down}
\begin{enumerate}
\item The user is notified that the server could not be reached.
\end{enumerate}
\subparagraph{Alternative Flow A1: Required Fields Ommitted}
\begin{enumerate}
\item The user does not fill in the required title or UPC fields.
\item Steps 3, 4, and 5 from the basic flow.
\item The server notifies the user that the required fields were not completed.
\end{enumerate}
\subparagraph{Alternative Flow A2: UPC not Unique}
\begin{enumerate}
\item The system notifies the user that the item cannot be added because the UPC is not unique.
\end{enumerate}

\paragraph{Post-conditions}
\begin{itemize}
\item The user is brought to the home page.
\item The added item is displayed under the ``Recent Changes'' list.
\end{itemize}

\paragraph{Extension Points}
None

\subsection{Basic Search for an Item}

\paragraph{Brief Description}
This use case shows how a basic search from any page  would proceed. This case will encompass the progression from the user entering the search term to the display of the search result(s).

\paragraph{Actors}
\begin{itemize}
\item User
\end{itemize}

\paragraph{Pre-conditions}
\begin{itemize}
\item The current page must have a basic search box.
\end{itemize}

\paragraph{Basic Flow of Events}
\begin{enumerate}
\item The user types a query in the search box.
\item The user clicks the search button.
\item The browser sends a request to the server. [A0]
\item The server queries the database for items entered into the system that match the query. [A1]
\item The server responds with all of the formatted and sorted results of the database query on the same page. [A2]
\item The user is directed to the Advanced Search page with the results of the basic search displayed at the bottom.
\end{enumerate}

\paragraph{Alternative Flows of Events}

\subparagraph{Alternative Flow A0: Empty Query}
\begin{enumerate}
\item The user remains on the current page.
\end{enumerate}

\subparagraph{Alternative Flow A1: Server is Down}
\begin{enumerate}
\item The user is notified that the server could not be reached.
\end{enumerate}

\subparagraph{Alternative Flow A2: No Matches}
\begin{enumerate}
\item The server responds with the message ``No results.''
\item The results page displays the message returned by the server.
\end{enumerate}

\paragraph{Post-conditions}
\begin{itemize}
\item The results page will display the response from the server or that the server was unreachable.
\item The search box will preserve the search term that was originally entered.
\item The results will be prioritized and sorted according to exact matching, then fuzzy matching.
\item The results will be matched using wildcards if they are present in the query.  This will replace the exact matching, and fuzzy matching will then be done on the query with the wildcard characters removed.
\end{itemize}

\paragraph{Extension Points}
\begin{itemize}
\item 5.4 Search Result Sorting
\item 5.5 View Details of an Item
\end{itemize}


\subsection{Advanced Search for an Item}

\paragraph{Brief Description}
This use case shows how a search from the advanced search page  would proceed. This case will encompass the progression from the user entering the search term to the display of the search result(s).

\paragraph{Actors}
\begin{itemize}
\item User
\end{itemize}

\paragraph{Pre-conditions}
\begin{itemize}
\item The current page must be the advanced search page.
\end{itemize}

\paragraph{Basic Flow of Events}
\begin{enumerate}
\item The user chooses a category from a dropdown menu. [A0]
\item The category selection is sent to the server.
\item The server responds with the attribute fields related to the category. [A1]
\item The category's attribute fields are dynamically added to the page.
\item The user optionally fills in the attribute, name, and UPC fields appropriately.
\item The user clicks the search button.
\item The browser sends a request to the server.
\item The server queries the database for items entered into the system that match the query. [A1]
\item The server responds with the formatted and sorted results of the database query. [A3]
\item The list of candidates for the search parameter(s) provided is dynamically added to the bottom part of the page.
\end{enumerate}

\paragraph{Alternative Flow of Events}

\subparagraph{Alternative Flow A0: No Category}
\begin{enumerate}
\item The user fills in the name and/or UPC.
\item Return to basic flow step 6.
\end{enumerate}

\subparagraph{Alternative Flow A1: Server is Down}
\begin{enumerate}
\item The user is notified that the server could not be reached.
\end{enumerate}

\subparagraph{Alternative Flow A2: Empty Query}
\begin{enumerate}
\item The user remains on the current page.
\end{enumerate}

\subparagraph{Alternative Flow A3: No Matches}
\begin{enumerate}
\item The server responds with the message ``No results.''
\item The results page displays the message returned by the server.
\end{enumerate}

\paragraph{Post-conditions}
\begin{itemize}
\item The results page will display the results of the search.
\item The search fields will preserve the search parameter(s) that were originally entered.
\item If appropriate, the results will be prioritized and sorted according to exact matching, then fuzzy matching.
\item The results will be matched using wildcards if they are present in the query.  This will replace the exact matching, and fuzzy matching will then be done on the query with the wildcard characters removed.
\item If completed, the category field will only use exact matching.
\end{itemize}

\paragraph{Extension Points}
\begin{itemize}
\item 5.4 Search Result Sorting
\item 5.5 View Details of an Item
\end{itemize}


\subsection{Search Result Sorting}

\paragraph{Brief Description}
This use case shows how changing the sorting of the items on the search results page would proceed. This case will encompass the progression from the user clicking the sort type link to the display of the search result(s) in the new order.

\paragraph{Actors}
\begin{itemize}
\item User
\end{itemize}

\paragraph{Pre-conditions}
\begin{itemize}
\item The current page must be the search results page.
\end{itemize}

\paragraph{Basic Flow of Events}
\begin{enumerate}
\item The user clicks on a sort type link (options are Name, UPC, Recently Created, and Recently Modified).
\item The browser sends a request to the server.
\item The server queries the database for items entered into the system that match the query. [A0]
\item The server responds with the formatted and sorted results of the database query. [A1]
\item The screen displays to the user a list of candidates for the search parameter(s) provided.
\end{enumerate}

\paragraph{Alternative Flow of Events}

\subparagraph{Alternative Flow A0: Server is Down}
\begin{enumerate}
\item The user is notified that the server could not be reached.
\end{enumerate}

\subparagraph{Alternative Flow A1: No Matches}
\begin{enumerate}
\item The server responds with the message ``No results.''
\item The results page displays the message returned by the server.
\end{enumerate}

\paragraph{Post-conditions}
\begin{itemize}
\item The results page will display the results of the search.
\item The search field(s) will preserve the search parameter(s) that were originally entered.
\item The search fields will be sorted by match method first (wildcard, exact, or fuzzy), then by the selected sort type.
\end{itemize}

\paragraph{Extension Points}
\begin{itemize}
\item 5.5 View Details of an Item
\end{itemize}


\subsection{View Details of an Item}
\paragraph{Brief Description}
This use case shows how a user would select an item from a list to view more detailed information about it.

\paragraph{Actors}
\begin{itemize}
\item User
\end{itemize}

\paragraph{Pre-conditions}
\begin{itemize}
\item The current page must have a link to the item which the user would like to view, as might happen in search results list or the recently modified items list.
\end{itemize}

\paragraph{Basic Flow of Events}
\begin{enumerate}
\item The user clicks on an item (in search results, the recent changes list, or any other place with items listed).
\item The browser queries the server for the item using the UPC code to specify the item.
\item The server responds with all the data stored about the item. [A0]
\item The browser takes the user to a new page with all the data about the item that the server sent.  It is displayed in elements labeled with ``Category:'', ``Size:'', etc. [A1]
\end{enumerate}

\paragraph{Alternative Flow of Events}

\subparagraph{Alternative Flow A0: Server is Down}
\begin{enumerate}
\item The user is notified that the server could not be reached.
\end{enumerate}

\subparagraph{Alternative Flow A1: Item not Found}
\begin{enumerate}
\item The page displays a message saying that the item could not be found.
\end{enumerate}

\paragraph{Post-conditions}
\begin{itemize}
\item The page will display all data about the item that the system currently has logged for the item
\end{itemize}

\paragraph{Extension Points}
\begin{itemize}
\item 5.6 Edit Details of an Item
\end{itemize}

\subsection{Edit Details of an Item}

\paragraph{Brief Description}
This use case shows how a user would modify an item that already exists in the server and save the edited/new field information for future viewing.

\paragraph{Actors}
\begin{itemize}
\item User
\end{itemize}

\paragraph{Pre-conditions}
\begin{itemize}
\item The current page must be an item's detailed description page.
\end{itemize}

\paragraph{Basic Flow of Events}
NOTE: The required fields cannot be unset after creation of the item.  This means that required fields may be changed to any valid and non-empty values.
\begin{enumerate}
\item The user clicks on the value of a field.
\item The page replaces the element with a text field containing the text that was in the element.
\item The user makes the desired changes to the value.
\item The user clicks outside of the field.
\item The browser sends a request to the server asking the value to be changed in the database.
\item The server changes the value in the database. [A0, A1]
\item The server responds that the value was changed.
\item The page changes the text field back into a div element that contains the updated value of the field.
\end{enumerate}

\paragraph{Alternative Flow of Events}

\subparagraph{Alternative Flow A0: Server is Down}
\begin{enumerate}
\item The user is notified that the server could not be reached.
\end{enumerate}

\subparagraph{Alternative Flow A1: Change Unsuccessful}
NOTE: This can happen when the database encounters an error or is not online.
\begin{enumerate}
\item The server responds that the value could not be changed.
\item The page notifies the user that the value could not be changed.
\item The user presses ``Ok'' in the message dialog.
\item The text field is focused again.
\end{enumerate}

\paragraph{Post-conditions}
\begin{itemize}
\item The data on the page will represent the updated state of the data in the system.
\end{itemize}

\paragraph{Extension Points}
None

\subsection{Generate Inventory Report}

\paragraph{Brief Description}
This use case shows the procedure for generating reports of the item inventory.

\paragraph{Actors}
\begin{itemize}
\item User
\end{itemize}

\paragraph{Pre-conditions}
\begin{itemize}
\item The current page must have a link to the ``Generate Report'' page.
\end{itemize}

\paragraph{Basic Flow of Events}
\begin{enumerate}
\item The user clicks the ``Generate Report'' link. [A0]
\item The user is brought to the ``Generate Report'' page.
\item The user chooses the type of report to generate.
\item The user chooses appropriate report fields for the respective option (the options are defined as Master or Category, which will be elaborated on near the completion).
\item The system generates the report accordingly.
\end{enumerate}

\paragraph{Alternative Flows of Events}

\subparagraph{Alternative Flow A0: Server is Down}
\begin{enumerate}
\item The user is notified that the server could not be reached.
\end{enumerate}

\paragraph{Post-conditions}
\begin{itemize}
\item After basic step 5, the user is displayed the report which is generated as specified.
\end{itemize}

\paragraph{Extension Points}
None

\section{Coding Standards}
\begin{itemize}
\item The code will use spaces, not tabs, to indent blocks
\item The code will use indentation increments of two spaces
\item The code will use underscores to separate words in variable names
\item The code will follow any other Ruby coding standards[14]
\item The folder structure for the product will be similar to that of the GasTracker project, a folder structure the client is familiar with
\end{itemize}

\section{Change Control}

\subsection{Receiving Requests}
If the client wishes to change the code, he will notify the team at a meeting.  Otherwise, he will fork the repository, make the changes himself, and submit a pull request.  If information is required for a change, it will be communicated either orally or via e-mail.  There will be no structured template for requesting changes, and more information will be requested if necessary.

\subsection{Managing Change and Source Control}
\begin{itemize}
\item Changes to sections carried over from previous milestones will be noted in a ``Deltas'' section, which has already been implemented in previous milestones
\item Changes to a milestone will be noted using the Git[13] change log
\item When content carried over from a previous milestone is changed, any milestones containing that content will be updated
\item Changes to anything in the repository will be noted with a detailed commit message
\item The source will be kept in a public repository on GitHub[13]
\end{itemize}

\section{Test Cases}
\label{test_case}

\paragraph{Test Case Syntax}
Each test case is divided into three sections:
\begin{itemize}
\item A ``Scenario Matrix'' section which describes each possible combination of flows (both basic and alternative) of a use case.
\item A ``Testing Matrix'' section which identifies distinct test cases defined by what conditions must be present in order to enact an expected result in response to the situation.
\item A ``Test Case to Scenario Matrix'' section which shows the correspondence between scenarios and test cases in order to help ensure completeness and traceability.


\subsection{Add an Item}
\textbf{NOTE:} Alternate Flow A0 will not be tested, since the server going offline is not an attribute of the system that can be measured.\label{flow}

\paragraph{Scenario Matrix}~\\ \\
\begin{tabular}{ c  c  c  c }
\hline
Scenario ID & Originating Flow & Alternate Flow & Next Alternate\label{scenario}\\
\hline
\hline
S0 & Basic &  & \\
\hline
S1 & Basic & A1 & \\
\hline
S2 & Basic & A2 & \\
\hline
S3 & Basic & A1 & A2\\
\hline
\end{tabular}\\
~\\
~\\
\paragraph{Testing Matrix}~\\ \\
\begin{tabular}{ p{0.8in}  p{0.7in}  p{0.7in}  p{3.3in} }
\hline
Test Case ID & Name Field & UPC Field & Expected Result\\
\hline
\hline
T0 & Present & Unique & The item is added successfully and the user is redirected to the main page with an updated Recent Changes list\\
\hline
T1 & Absent & Unique & The user is notified that a required field was omitted\\
\hline
T2 & Present & Absent & The user is notified that a required field was omitted\\
\hline
T3 & Absent & Absent & The user is notified that a required field was omitted\\
\hline
T4 & Present & Non-unique & The user is notified that the UPC is not unique\\
\hline
T5 & Absent & Non-unique & The user is notified that a required field was omitted and that the UPC is not unique\\
\hline
\end{tabular}\\
~\\
~\\
Present - a non-empty input\\
Absent - an empty input\\
Unique - a UPC code that is exclusive to the item\\
Non-unique - a UPC code that is no exclusive to the item
\paragraph{Test Case to Scenario Matrix}~\\ \\
\begin{tabular}{ | c || c | c | c | c | }
\hline
   & S0 & S1 & S2 & S3 \\
\hline
\hline
T0 & X  &    &    &    \\
\hline
T1 &    & X  &    &    \\
\hline
T2 &    & X  &    &    \\
\hline
T3 &    & X  &    &    \\
\hline
T4 &    &    & X  &    \\
\hline
T5 &    &    &    & X  \\
\hline
\end{tabular}

\subsection{Basic Search for an Item}
\textbf{NOTE:} Alternate Flow A1 will not be tested, since the server going offline is not an attribute of the system that can be measured.

\paragraph{Scenario Matrix}~\\ \\
\begin{tabular}{ c  c  c }
\hline
Scenario ID & Originating Flow & Alternate Flow\\
\hline
\hline
S4 & Basic &  \\
\hline
S5 & Basic & A0 \\
\hline
S6 & Basic & A2 \\
\hline
\end{tabular}\\
~\\
~\\
\paragraph{Testing Matrix}~\\ \\
\begin{tabular}{ p{0.8in}  p{0.75in}  p{0.5in}  p{3in} }
\hline
Test Case ID & Query Field & Matches & Expected Result\\
\hline
\hline
T6 & Present & Present & The results of the query are displayed on the page\\
\hline
T7 & Absent & & The user remains on the same page\\
\hline
T8 & Present & Absent & The text ``No results'' is displayed on the page\\
\hline
\end{tabular}\\
~\\
~\\
Present - a non-empty input\\
Absent - an empty input
\paragraph{Test Case to Scenario Matrix}~\\ \\
\begin{tabular}{ | c || c | c | c | }
\hline
   & S4 & S5 & S6 \\
\hline
\hline
T6 & X  &    &    \\
\hline
T7 &    & X  &    \\
\hline
T8 &    &    & X  \\
\hline
\end{tabular}


\subsection{Advanced Search for an Item}
\textbf{NOTE:} Alternate Flow A1 will not be tested, since the server going offline is not an attribute of the system that can be measured.

\paragraph{Scenario Matrix}~\\ \\
\begin{tabular}{ c  c  c  c }
\hline
Scenario ID & Originating Flow & Alternate Flow & Next Alternate \\
\hline
\hline
S7 & Basic &  & \\
\hline
S8 & Basic & A0 & \\
\hline
S9 & Basic & A0 & A2 \\
\hline
S10 & Basic & A0 & A3 \\
\hline
S11 & Basic & A3 &  \\
\hline
\end{tabular}\\
~\\
~\\
\paragraph{Testing Matrix}~\\ \\
\begin{tabular}{ p{0.8in}  p{0.5in} p{0.7in}  p{0.5in}  p{3in} }
\hline
Test Case ID & Category Field & Name/UPC Field & Matches & Expected Result\\
\hline
\hline
T9 & Present &  & Present & The results of the query are displayed on the page\\
\hline
T10 & Absent & Present & Present & The results of the query are displayed on the page\\
\hline
T11 & Absent & Absent &  & The user remains on the current page\\
\hline
T12 & Absent & Present & Absent & The text ``No results'' is displayed on the page\\
\hline
T13 & Present &  & Absent & The text ``No results'' is displayed on the page\\
\hline
\end{tabular}\\
~\\
~\\
Present - a non-empty input\\
Absent - an empty input
\paragraph{Test Case to Scenario Matrix}~\\ \\
\begin{tabular}{ | c || c | c | c | c | c | }
\hline
    & S7  & S8  & S9  & S10 & S11 \\
\hline
\hline
T9  &  X  &     &     &     &     \\
\hline
T10 &     &  X  &     &     &     \\
\hline
T11 &     &     &  X  &     &     \\
\hline
T12 &     &     &     &  X  &     \\
\hline
T13 &     &     &     &     &  X  \\
\hline
\end{tabular}

\subsection{Search Result Sorting}
\textbf{NOTE:} Alternate Flow A0 will not be tested, since the server going offline is not an attribute of the system that can be measured.
\paragraph{Scenario Matrix}~\\ \\
\begin{tabular}{ c  c }
\hline
Scenario ID & Originating Flow\\
\hline
\hline
S12 & Basic\\
\hline
\end{tabular}\\
~\\
~\\
\paragraph{Testing Matrix}~\\ \\
\begin{tabular}{ p{0.8in}  p{1.1in}  p{3.6in} }
\hline
Test Case ID & Sort Type Field & Expected Result\\
\hline
\hline
T14 & Name & The search results are sorted by the Name field\\
\hline
T15 & UPC & The search results are sorted by the UPC field\\
\hline
T16 & Recently Created & The search results are sorted by the Recently Created field\\
\hline
T17 & Recently Modified & The search results are sorted by the Recently Modified field\\
\hline
\end{tabular}\\
~\\
~\\
Name - a required field that serves as a label; sorting for this field is alphanumeric (ascending or descending)\\
UPC - a required field that serves as an unique identifier; sorting for this field is numeric (ascending or descending)\\
Recently Created - an automatically generated field that denotes the timestamp of when an item was added; sorting for this field is based on date and time (ascending or descending)\\
Recently Modified - an automatically generated field that denotes the timestamp of when an item was added or changed; sorting for this field is based on date and time (ascending or descending)
\paragraph{Test Case to Scenario Matrix}~\\ \\
\begin{tabular}{ | c || c | }
\hline
    & S12 \\
\hline
\hline
T14 &  X  \\
\hline
T15 &  X  \\
\hline
T16 &  X  \\
\hline
T17 &  X  \\
\hline
\end{tabular}

\subsection{View Details of an Item}
\textbf{NOTE:} Alternate Flow A0 will not be tested, since the server going offline is not an attribute of the system that can be measured.

\paragraph{Scenario Matrix}~\\ \\
\begin{tabular}{ c  c }
\hline
Scenario ID & Originating Flow\\
\hline
\hline
S13 & Basic\\
\hline
\end{tabular}\\
~\\
~\\
\paragraph{Testing Matrix}~\\ \\
\begin{tabular}{ p{0.8in}  p{3.4in} }
\hline
Test Case ID & Expected Result\\
\hline
\hline
T18 & The user is taken to a page showing all of the item details\\
\hline
\end{tabular}\\
~\\
~\\
\paragraph{Test Case to Scenario Matrix}~\\ \\
\begin{tabular}{ | c || c | }
\hline
    & S13 \\
\hline
\hline
T18 &  X  \\
\hline
\end{tabular}

\subsection{Edit Details of an Item}
\textbf{NOTE:} Alternate Flow A0 will not be tested, since the server going offline is not an attribute of the system that can be measured.
\paragraph{Scenario Matrix}~\\ \\
\begin{tabular}{ c  c  c }
\hline
Scenario ID & Originating Flow & Alternate Flow\\
\hline
\hline
S14 & Basic & \\
\hline
S15 & Basic & A1\\
\hline
\end{tabular}\\
~\\
~\\
\paragraph{Testing Matrix}~\\ \\
\begin{tabular}{ p{0.8in}  p{0.9in}  p{0.9in}  p{2.8in}}
\hline
Test Case ID & Name Field & UPC Field & Expected Result\\
\hline
\hline
T19 & Unchanged & Unchanged & The user is shown the item details page with the modified information\\
\hline
T20 & Valid Change & Unchanged & The user is shown the item details page with the modified information\\
\hline
T21 & Unchanged & Valid Change & The user is shown the item details page with the modified information\\
\hline
T22 & Valid Change & Valid Change & The user is shown the item details page with the modified information\\
\hline
T23 & Invalid Change & Unchanged & A pop-up with `Invalid Information' is shown\\
\hline
T24 & Invalid Change & Valid Change & A pop-up with `Invalid Information' is shown\\
\hline
T25 & Invalid Change & Invalid Change & A pop-up with `Invalid Information' is shown\\
\hline
T26 & Unchanged & Invalid Change & A pop-up with `Invalid Information' is shown\\
\hline
T27 & Valid Change & Invalid Change & A pop-up with `Invalid Information' is shown\\
\hline
\end{tabular}\\
~\\
~\\
Unchanged - the field was not modified\\
Valid Change (relative to Name Field) - the field was modified to be non-empty\\
Valid Change (relative to UPC Field) - the field was modified to be non-empty and unique\\
Invalid Change (relative to Name Field) - the field was modified to be empty\\
Invalid Change (relative to UPC Field) - the field was modified to be empty or non-unique
\paragraph{Test Case to Scenario Matrix}~\\ \\
\begin{tabular}{ | c || c | c | }
\hline
    & S14 & S15 \\
\hline
\hline
T19 &  X  &     \\
\hline
T20 &  X  &     \\
\hline
T21 &  X  &     \\
\hline
T22 &  X  &     \\
\hline
T23 &     &  X  \\
\hline
T24 &     &  X  \\
\hline
T25 &     &  X  \\
\hline
T26 &     &  X  \\
\hline
T27 &     &  X  \\
\hline
\end{tabular}

\subsection{Generate Inventory Report}
\textbf{NOTE:} Alternate Flow A0 will not be tested, since the server going offline is not an attribute of the system that can be measured.

\paragraph{Scenario Matrix}~\\ \\
\begin{tabular}{ c  c }
\hline
Scenario ID & Originating Flow \\
\hline
\hline
S16 & Basic \\
\hline
\end{tabular}\\
~\\
~\\
\paragraph{Testing Matrix}~\\ \\
\begin{tabular}{ p{0.8in}  p{2.2in} }
\hline
Test Case ID & Expected Result\\
\hline
\hline
T28 & The report is displayed on the page\\
\hline
\end{tabular}\\
~\\
~\\
\paragraph{Test Case to Scenario Matrix}~\\ \\
\begin{tabular}{ | c || c | }
\hline
    & S16  \\
\hline
\hline
T28 &  X  \\
\hline

\end{tabular}
\section{Index and Glossary}
\textbf{Assigned to}: Developer ultimately responsible for the implementation of a feature (\pageref{feature}).\\ \\
\textbf{Effort}: Expectation of the resources and time consumed for a feature (\pageref{feature}).\\ \\
\textbf{Exact Matching}: Valid search candidates are determined by strict equality with the search input(s) (\pageref{feature}).\\ \\
\textbf{Feature}: A system capability that fulfills a user need (\pageref{feature}).\\ \\
\textbf{Flow}: A particular series of ordered steps through a sequence of events in a use case (\pageref{flow}).\\ \\
\textbf{Fuzzy Matching}: Valid search candidates are determined based on similarity to the original search input(s) (\pageref{feature}).\\ \\
\textbf{Post-condition}: An expectation of the system's state after the events in a use case occur (\pageref{post_cond}).\\ \\
\textbf{Pre-condition}: An assumption about the state of a system before a use case's events are followed (\pageref{pre_cond}).\\ \\
\textbf{Priority}: Description of how essential a feature is to the project (\pageref{feature}).\\ \\
\textbf{REST}: Stands for ``Representational state transfer;'' is a style software architecture often used in servers for simplicity and scalability. Amongst other aspects of REST, it is stateless. This means each request sent from the client of the server that uses REST to the server does not require any information from any previous requests. All information required to fulfill the request must be sent with the request (\pageref{rest}).\\ \\
\textbf{Risk}: Probability that a feature will instigate delays in the project (\pageref{feature}).\\ \\
\textbf{Scenario}: A possible permutation of a set of flows that lead from one flow to another (\pageref{scenario}).\\ \\
\textbf{Stability}: Likelihood that the understanding of a feature will change (\pageref{feature}).\\ \\
\textbf{Target Release}: Expected release iteration of a testable feature (\pageref{feature}).\\ \\
\textbf{Test Case}: Specifies what the expected result of a series of actions will have in a system (\pageref{test_case}).\\ \\
\textbf{Universal Product Code (UPC)}: a specific kind of barcode assign uniquely to each item in the server (\pageref{upc}).\\ \\
\textbf{Use case}: A description of the steps performed by a user and system that leads the user towards a useful outcome (\pageref{use_case}).\\ \\
\textbf{Wildcard Matching}: Valid search candidates are determined based on adherence to the user's given search pattern (\pageref{feature}).\\ \\

\section{References}
\hangindent=1.4cm
\textbf{(1)} Leffingwell, Dean, and Don Widrig.
\emph{Managing Software Requirements: a Use Case Approach}.
Addison-Wesley, Boston,
2nd Edition,
2003.\\

\noindent\hangindent=1.4cm
\textbf{(2)} ``Ruby 1.9.2''
\emph{Download Ruby.} Ruby. Web.  6 October 2011. \\

\noindent\hangindent=1.4cm
\textbf{(3)} ``Sinatra 1.3.0''
\emph{Sinatra: Documentation.} Sinatra. Web.  6 October 2011.\\

\noindent\hangindent=1.4cm
\textbf{(4)} ``Ubuntu 11.04''
\emph{Download Ubuntu.} Ubuntu. Web.  6 October 2011.\\

\noindent\hangindent=1.4cm
\textbf{(5)} ``SQLite 3.7.8''
\emph{SQLite Download Page.} SQLite. Web.  6 October 2011.\\

\noindent\hangindent=1.4cm
\textbf{(6)} ``Cucumber 1.1.0''
\emph{cucumber/cucumber.} Cucumber. Web.  6 October 2011.\\

\noindent\hangindent=1.4cm
\textbf{(7)} ``RSpec 2.6.0''
\emph{RSpec Documentation.} Relish. Web.  6 October 2011.\\

\noindent\hangindent=1.4cm
\textbf{(8)} ``DataMapper 1.1.0''
\emph{DataMapper - Documentation.} DataMapper. Web.  6 October 2011.\\

\noindent\hangindent=1.4cm
\textbf{(9)} ``Google Chrome 14.0.8'' 
\emph{About Google Chrome.} Google. Web.  7 October 2011.\\

\noindent\hangindent=1.4cm
\textbf{(10)} ``Firefox 7.0.1''
\emph{Mozilla Firefox Web Browser - Free Download.} Mozilla. Web.  7 October 2011.\\

\noindent\hangindent=1.4cm
\textbf{(11)} ``Apache 2.2''
\emph{Apache HTTP Server Version 2.2 Documentation - Apache HTTP Server.} Apache. Web.  7 October 2011.\\

\noindent\hangindent=1.4cm
\textbf{(12)} ``BSD''
\emph{The FreeBSD Copyright.} The FreeBSD Project. Web. 18 October 2011.\\

\noindent\hangindent=1.4cm
\textbf{(13)} ``GitHub''
\emph{GitHub - Social Coding.} GitHub, Inc. Web. 28 October 2011.\\

\noindent\hangindent=1.4cm
\textbf{(14)} ``The Unofficial Ruby Usage Guide'' \emph{The Unofficial Ruby Usage Guide.} Caliban. 28 October 2011.\\

\end{document}